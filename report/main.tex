\documentclass{article}
\usepackage{bm}
\usepackage{harpoon}
\usepackage{listings}
\usepackage[left=2cm, right=2cm]{geometry}
\usepackage{setspace}
\usepackage{bm}
\usepackage{cmap}
\usepackage{cite}
\usepackage{float}
\usepackage{stmaryrd}
\usepackage{amsthm}
\usepackage{amsmath}
\usepackage{amssymb}
\usepackage{setspace}
\usepackage{enumerate}
\usepackage{indentfirst}
\usepackage{fontspec}
\usepackage{booktabs}
\usepackage{tikz}
\usepackage{subfigure}
\usepackage{graphicx}
\usetikzlibrary{arrows,decorations}
\usepackage[colorlinks, linkcolor=red]{hyperref}
\usepackage[indentfirst]{xeCJK}
\usepackage{rerunfilecheck}
\usepackage[dvipdfmx]{movie15_dvipdfmx}
\setCJKmainfont[BoldFont={SimHei},ItalicFont={KaiTi}]{SimSun}
\setCJKsansfont{KaiTi}

\setCJKfamilyfont{KaiTi}{楷体}
\newcommand{\kai}{\CJKfamily{KaiTi}}

\allowdisplaybreaks
\newfontfamily\Courier{Courier New}
\renewcommand{\figurename}{图}
\renewcommand{\tablename}{表}

\def\ack{\section*{Acknowledgements}%
  \addtocontents{toc}{\protect\vspace{6pt}}%
  \addcontentsline{toc}{section}{Acknowledgements}%
}

\lstset{
	language=C++,
	tabsize=4,
	numbers=left,
	breaklines=tr,
	extendedchars=false
	xleftmargin=0em,
	xrightmargin=0em,
	aboveskip=1em,
	numberstyle=\small\Courier,
    basicstyle=\small\Courier
}

\renewenvironment{abstract}{
\subsection*{\textbf{摘~~要}}
\kai
}{}

\renewcommand{\contentsname}{\large 目~~录}

\newenvironment{keywords}{
\large\noindent\textbf{关键词}\,\,
}{\kai}
\begin{document}
	\title{\textbf{基于OpenGL$^{\text{TM}}$的物理引擎实现及应用}}
	\author{杨嘉成、黎金宁、秋闻达\\谭博文、胥拿云、许臻佳}
	\maketitle
	\begin{abstract}
		OpenGL$^{\text{TM}}$是行业领域中最为广泛接纳的2D/3D图形API\cite{ref2},其自诞生至今已催生了各种计算机平台及设备上的数千优秀应用程序。OpenGL$^{\text{TM}}$是独
		立于视窗操作系统或其它操作系统的,亦是网络透明的。在包含CAD、内容创作、能源、娱乐、游戏开发、制造业、制药业及虚拟现实等行业领域中,OpenGL$^{\text{TM}}$帮助程序员
		实现在PC、工作站、超级计算机等硬件设备上的高性能、极具冲击力的高视觉表现力图形处理软件的开发。本文作者编写了一种建立在OpenGL$^{\text{TM}}$上的物理引擎
		\cite{ref1},并且用该物理引擎探究了热力学和三体问题等物理问题,本文将从该物理引擎的实现出发,描述整个引擎的算法,并且将给出模拟实验结果。
	\end{abstract}
	\tableofcontents
	\newpage
	\section{OpenGL$^{\text{TM}}$概述}
	\subsection{工作机制}
	\indent 假设现在我们有一副物理图像需要显示在电脑屏幕上,那么首先我们会将这份物理图像的细节传输到OpenGL$^{\text{TM}}$的缓冲区中,之后OpenGL$^{\text{TM}}$根据缓冲区
	中数据,进行渲染工作(即将缓冲区中的数据转化为图像),最后通过交换缓存区(glutSwapBuffers)操作将图像显示在屏幕上。\par
	值得注意的是,这个工作机制决定了之后的引擎的工作机制,即引擎应该给出每个时刻的物理图像,而非只给出关键的事件点,OpenGL$^{\text{TM}}$只能从一个时刻过渡到和它紧邻的下
	一个时刻,而不能跳跃地从一个事件过渡到另外一个事件。
	具体步骤如下图所示:
	\begin{figure}[!htbp]
		\centering
		\includegraphics[width=15.5cm]{graph1.jpg}
	\end{figure}
	\subsection{评价}
	根据之后的实验结果,我们发现OpenGL$^{\text{TM}}$具有如下几个特点(优点/缺点):
	\begin{itemize}
		\item 相比其他的Graphics API(例如Microsoft DirectX),OpenGL$^{\text{TM}}$兼容性极强,同一份代码基本能够同时在Windows和Linux操作系统上编译并且达到基本相同的效果;
		\item OpenGL$^{\text{TM}}$与C++能够紧密结合,将C++的运算速度这个优点和OpenGL$^{\text{TM}}$强大的渲染机制结合起来,增强了计算速度。和WebGL$^{\text{TM}}$与JavaScript的组合相
		比,OpenGL$^{\text{TM}}$和C++的组合显然更具有竞争力;
		\item OpenGL$^{\text{TM}}$操作图像相对麻烦,同时在绘制轨迹的时候只能保存下轨迹之后再逐步绘制,一定程度上影响的绘制图像的效率;
		\item OpenGL$^{\text{TM}}$是过程性的API,因而OpenGL$^{\text{TM}}$不支持面向对象编程,一定程度上影响了编写程序时的分工与代码可读性;
		\item C++不能很好地将图表纳入程序中,因此实验的具体结果只能在模拟计算之后再进行分析。
	\end{itemize}
	\newpage
	\section{驱动引擎}
	\subsection{分子动力学引擎}
	\subsubsection{分子动力学方法}
	\indent 分子动力学法(MD方法,Molecular Dynamics Method),是用计算机模拟的一种,是调查物质诸性质时候使用的手法之一。根据在计算机中每时每刻的追踪全部的粒子的运动的规
	律,导出物质全体的性质这就是分子动力学法。\par
	\indent 分子动力学法是要严格求解每个粒子的运动方程,通过分析系统来确定粒子的运动状态。MD方法一般认为粒子服从牛顿运动规律,粒子所受到的作用力通过粒子间相互作用势计算,
	因此粒子的初始条件和运动方程中的受力状况一旦确定,系统就可被精确求解。
	\subsubsection{评价}
	\begin{itemize}
		\item 编程复杂度较低,基本上只需要给出运动方程就能够模拟;
		\item 灵活性高,在之后将会看到分子动力学方法作为了大部分引擎的内核;
		\item 计算速度较慢,因为每次显示都需要重新计算每个分子的受力情况;
		\item 模拟精度较差,如果采用的硬球势,那么就很有可能会出现两个分子对穿而过的情况;
	\end{itemize}
	\subsection{事件驱动引擎}
	如果不考虑$n$个气体分子的其他相互作用(即只有碰撞),那么考虑我们现在有一个指针表示当前显示的图像所对应的时刻,如果当前时刻小于下一个碰撞,那么所有气体分子都向它所在的那个
	方向运动一个极小的时间$\mathrm{d}t$,否则我们按照动量守恒改变每个例子的速度
	\subsubsection{碰撞判定}
	\subsubsection*{球与球的碰撞}
	假设有两个小球,中心位置分别是$\mathbf{r_1}(x_1,y_1)$,$\mathbf{r_2}(x_2,y_2)$,速度分别是$\mathbf{v_1}(v_{x_1},v_{y_1})$,$\mathbf{v_2}(v_{x_2},v_{y_2})$,半径
	分别是$r_1$,$r_2$。以球1为参考系,将球1中心位置作为新的坐标系的原点,则球2的中心坐标变为$\mathbf{r_2'}(x_2-x_1,y_2-y_1)$,速度变
	为$\mathbf{v_2'}(v_{x_2}-v_{x_1},v_{y_2}-v_{x_1})$。考虑到相撞即为球心距为$d_1+d_2$,所以可以将球2看作点,球1的半径变
	为$d_1'=d_1+d_2$。得到方程:
	\[|\mathbf{r_2'}+t\mathbf{v_2'}|=r_1'\]
	即:
	\[[x_2-x_1+t(v_{x_2}-v_{x_1})]^2+[y_2-y_1+t(v_{y_2}-v_{y_1})]^2=(d_1+d_2)^2\]
	解方程即可得到碰撞时间$t$,若没有非负实数解,那么碰撞不可能发生。
	\subsubsection*{球与墙的碰撞}
	球与墙的碰撞分为球与$x$方向墙的碰撞和球与$y$方向墙的碰撞。两种情况计算方法类似,这里以x方向墙为例:
	假设墙位于$x=x_0$;球的中心位置是$\mathbf{r}(x,y)$,速度是$\mathbf{v}(v_x,v_y)$,半径是$d$,如果球在时间$t$之后和墙相碰,那么显然有:
	\[|x_0-(x+tv_x)|=d\]
	解得:
	\[t=\frac{x_0-x}{v_x}-\frac{d}{|v_x|}\]
	解方程即可得到碰撞时间$t$,若没有非负实数解,那么碰撞不可能发生。
	\subsubsection{碰撞结果}
	如果只考虑弹性碰撞\cite{ref3},设两个球的位矢分别是$\mathbf{r_1}$和$\mathbf{r_2}$,质量分别是$m_1$和$m_2$,速度分别是$\mathbf{v_1}$和$\mathbf{v_2}$,先将速度分解为
	平行于$\mathbf{r} = \mathbf{r_2} - \mathbf{r_1}$的方向和垂直于$\mathbf{r}$的方向,设为$\mathbf{v_{1_\sslash}}$、$\mathbf{v_{1_\perp}}$和
	$\mathbf{v_{2_\sslash}}$、$\mathbf{v_{2_\perp}}$,其中垂直于$\mathbf{r}$方向的速度在碰撞过程中不变,平行于$\mathbf{r}$方向的速度的改变情况和一维碰撞类似,即:
	\[\mathbf{v_{1_\sslash}'} = \frac{(m_1 - m_2)\mathbf{v_{1_\sslash}}+2m_2\mathbf{v_{2_\sslash}}}{m_1 + m_2}\]
	\[\mathbf{v_{2_\sslash}'} = \frac{(m_2 - m_1)\mathbf{v_{2_\sslash}}+2m_1\mathbf{v_{1_\sslash}}}{m_1 + m_2}\]
	因此碰撞之后的速度为:
	\[\mathbf{v_1'} = \frac{(m_1 - m_2)\mathbf{v_{1_\sslash}}+2m_2\mathbf{v_{2_\sslash}}}{m_1 + m_2} + \mathbf{v_{1_\perp}}\]
	\[\mathbf{v_2'} = \frac{(m_2 - m_1)\mathbf{v_{2_\sslash}}+2m_1\mathbf{v_{1_\sslash}}}{m_1 + m_2} + \mathbf{v_{2_\perp}}\]
	\subsubsection{碰撞事件维护}
	\subsubsection*{基本想法}
	\indent 考虑$n$个气体分子有各自的初始位置,速度,质量,它们最多两两之间发生碰撞,同时每个气体分子会和上下左右四个边界发生碰撞,因此最多有$O(n^2)$个碰撞事件。
	记$f[x][y]$表示气体分子$x$和气体分子$y$发生碰撞的时间($1 \leq x < y \leq n$,如果不发生碰撞就记为$+\infty$)。记$g[x][p]$表示气体分子$x$和上下左右边界碰撞的时间
	($1 \leq x \leq n, 1\leq p \leq 4$分别表示上下左右,如果不发生碰撞就记为$+\infty$)。\par
	最早发生的碰撞就是$f,g$中最小的时间所对应的碰撞,碰撞后重新计算每个气体分子的位置、速度,然后计算它们之间的碰撞事件。这样每次找到下一次碰撞事件的复杂度是$O(n^2)$。
	\subsubsection*{优化1}
	因为每次是得到最小值,所以可以用优先队列进行优化,这样每次找最小值就不需要全部扫描$f,g$,而是可以在$O(\log n)$的时间内得到最小值。\par
	但是由于取出最小值后要重新计算$f,g$,并且重新构建优先队列,因此总体复杂度还是$O(n^2)$,几乎没有改进,但是这种方法却启示了我们进行进一步优化。
	\subsubsection*{优化2}
	对于一次碰撞(气体分子$x$和气体分子$y$发生碰撞),除了$x,y$的其他气体分子之间的碰撞并没有发生改变,因此发生一次碰撞之后并没有必要把所有气体分子全部重算,而只要重算发生碰撞的原
	子($x,y$)和其他气体分子之间的碰撞,这样修改量就只有$O(n)$个,总的复杂度优化到了$O(n\log n)$。\par
	具体步骤如下:
	\begin{enumerate}
		\item 取出优先队列的队头元素(气体分子$x$和气体分子$y$发生碰撞);
		\item 将$x,y$和其他气体分子的碰撞从优先队列中删除;
		\item 计算$x,y$碰撞后的速度和原始位置;
		\item 计算$x,y$和其他气体分子的碰撞,加入到优先队列中。
	\end{enumerate}
	\subsubsection{评价}
	\begin{itemize}
		\item 模拟速度快,因为只需要处理关键的碰撞情况;
		\item 只能模拟含碰撞的情况,如果加上了重力场或者是其他外力时就不能维护了;
		\item 编程复杂度较大,这个优先队列不能使用C++ STL中的priority\_queue来实现,因为我们可能会在该优先队列中删除某个中间的元素,因此我们必须自己实现一个支持
		删除操作的优先队列。
	\end{itemize}
	\newpage
	\section{应用}
	\subsection{热力学模拟实验}
	\subsubsection{布朗运动}
	\subsubsection*{实验原理}
	被分子撞击的悬浮微粒做无规则运动的现象叫做布朗运动。布朗运动是将看起来连成一片的液体,在高倍显微镜下看其实是由许许多多分子组成的。液体分子不停地做无规则的运动,不断地随机撞击悬浮微粒。当悬浮的微粒足够小的时候,由于受到的来自各个方向的液体分子的撞击作用是不平衡的。在某一瞬间,微粒在另一个方向受到的撞击作用超强的时候,致使微粒又向其它方向运动,这样就引起了微粒的无规则的运动,即布朗运动。 \\
	对于布朗运动,爱因斯坦的结论是:
	\[ \langle r^2\rangle = \frac{6kT}{\mu} t \]
	\subsubsection*{模拟过程}
	使用我们的引擎模拟大量微粒的无规则行走和碰撞。可以在其中得到任意时刻各个微粒的位置。从而计算出方均位移。将数据导入Origin软件,可以看出从开始一段时间后(下图中$150s$后),方均位移与时间成线性关系。从而在某个方面验证了爱因斯坦关于布朗运动所得的结论。
	\begin{figure}[H]
		\centering
		\subfigure[$<r^2> - t$图像] {
			\label{fig:a}
			\includegraphics[totalheight=10cm]{graph2.jpg}
		}
	\end{figure}
	\subsubsection{熵增实验}
	\subsubsection*{实验原理}
	热力学中,熵变定义为:
	\[\mathrm{d}S = \frac{\mathrm{\delta}Q}{T}\]
	对于等温过程中:
	\[S - S_0 = -vR\ln\frac{P}{P_0}\]
	其中$S_0$和$S$表示起始和末态的熵,在实验中对一个稳定体系加一些扰动,然后在等温过程下让系统趋于平衡。
	由于压强和与墙壁的碰撞次数有关,因此这里可以用单位时间内和容器壁的碰撞次数来对熵进行估计。
	\subsubsection*{模型1}
	左侧有一些随机运动的气体分子,右侧是排列整齐的静止的气体分子:
	\begin{figure}[H]
		\subfigure[熵变模型1] { \label{fig:b}
		      \includemovie[
		      autoplay,
		      controls,
		      %repeat=3,
		      text={\large(Loading video)}
		      ]{15cm}{8cm}{shang1.mp4}
		}
	\end{figure}
	\subsubsection*{模型2}
	所有气体分子排列整齐,少数气体分子具有随意初速度,其他气体分子初始静止:
	\begin{figure}[H]
		\subfigure[熵变模型2]{
			\label{fig:b}
		    \includemovie[
		    	autoplay,
		    	controls,
		    	%repeat=3,
				text={\large(Loading video)}
			]{15cm}{8cm}{shang2.mp4}
		}
	\end{figure}
	\subsubsection*{实验结果及分析}
	每个模型进行三次试验,每次试验统计连续的8个时间间隔内气体分子和容器壁碰撞的次数如下表,其中$\mathrm{u}$表示单位时间:
	\begin{table}[H]
		\centering
		\caption{\label{tab:a}熵的变化(模型1)}
		\begin{tabular}{ccccccccc}
			\toprule
			试验次数 & $1\mathrm{u}$ & $2\mathrm{u}$ & $3\mathrm{u}$ & $4\mathrm{u}$ & $5\mathrm{u}$ & $6\mathrm{u}$ & $7\mathrm{u}$ & $8\mathrm{u}$\\
			\midrule
			1 &   8  & 25 & 26 & 43 & 46 & 37 & 55 & 53 \\
			2 &   13 & 37 & 56 & 65 & 62 & 71 & 61 & 52 \\
			3 &   14 & 15 & 24 & 28 & 31 & 29 & 38 & 31 \\
			\bottomrule
		\end{tabular}
	\end{table}
	\begin{table}[H]
		\centering
		\caption{\label{tab:a}熵的变化(模型2)}
		\begin{tabular}{ccccccccc}
			\toprule
			试验次数 & $1\mathrm{u}$ & $2\mathrm{u}$ & $3\mathrm{u}$ & $4\mathrm{u}$ & $5\mathrm{u}$ & $6\mathrm{u}$ & $7\mathrm{u}$ & $8\mathrm{u}$\\
			\midrule
			1 &   21 & 28 & 51 & 62 & 122 & 144 & 143 & 128 \\
			2 &   18 & 15 & 63 & 78 & 97 & 157 & 155 & 128 \\
			3 &   23 & 23 & 30 & 66 & 107 & 143 & 123 & 129 \\
			\bottomrule
		\end{tabular}
	\end{table}
	通过两个模型都可以看出一开始气体分子和容器壁碰撞较少,之后碰撞逐渐增多,最后趋于稳定(有少许波动)。相对应的系统的熵从一个较小值逐渐增大,最后趋于平衡。
	\subsubsection{理想气体状态方程的验证}
	\subsubsection*{原理简述}
	理想气体状态方程,又称理想气体定律、普适气体定律,是描述理想气体在处于平衡态时,压强、体积、物质的量、温度间关系的状态方程。它建立在玻义耳-马略特定律、查理定律、盖-吕萨克定律等经验定律上。\\
	其方程为\[PV = NRT\]\\这个方程有$4$个变量:$P$是指理想气体的压强,$V$为理想气体的体积,$N$表示气体物质的量,而$T$则表示理想气体的热力学温度;还有一个常量$R$ 为理想气体常数。可以看出,此方程的变量很多。因此此方程以其变量多、适用范围广而著称,对常温常压下的空气也近似地适用。
	\subsubsection*{模拟过程}
	我们模拟一个充斥着许多无规则运动的气体分子的体积固定形态固定的容器,我们为每个粒子赋予随机的初始位置和速度,从而希望能够模拟出理想气体的状态。基于我们的物理引擎,体积$V$为定值,粒子数量$N$为定值,我们设定了不同的体系平均初始动能,代表温度$T$,统计一定时间内粒子与容器壁碰撞的动量变化,以此表示对容器壁的压力,由于容器体积与形态固定,这个物理量可以线性表示出体系的压强。\\
	对参数选择的说明:我们将粒子假设为硬球体,一切碰撞为完全弹性碰撞,整个体系动能保持不变,因此任意时刻体系平均动能等于初始设定的体系平均动能,从而我们可以使用体系的初始平均动能表示温度$T$;考虑粒子与容器壁碰撞时的动量变化$m \Delta v = F \Delta t$,$F$即为对容器壁的压力,由于粒子为硬球体,容器壁为硬边界,碰撞的时间$\Delta t$可视为同一极小量,从而动量变化可以以一个相同的比例系数线性表示压力,考虑容器的体积与形态固定,此物理量可线性表示压强$P$。
	\begin{figure}[H]
		\centering
		\subfigure[等容绝热理想气体P-T图] {
			\label{fig:a}
			\includegraphics[totalheight=8.5cm]{graph.jpg}
		}
	\end{figure}
	\subsection{三体问题模拟实验}
	\subsubsection{实验原理}
	在三体问题中,我们采用和分子动力学方法差不多的模拟方法,即计算每颗星球在每个时刻的受力,并且假设这颗星球在一个极短的时间内所受的力不变,然后以此来计算下一个时刻的
	各个物理参量。下面列出每个星球的运动方程:
	\[\mathbf{a_i} = \frac{1}{m_i}\sum_{k \neq i} G\frac{m_im_k}{\lVert \mathbf{r_k} - \mathbf{r_i} \rVert^3}(\mathbf{r_k} - \mathbf{r_i})\]
	\subsubsection{实验结果及分析}
	下面是在灰尘的坐标差异仅有0.1的情况下所得到的灰尘的运动轨迹(其中红色和蓝色的球表示两个质量极大的星球,黑色的点表示一粒灰尘,黑色的线表示灰尘的轨迹,红色的线
	表示位置矢量):
	\begin{figure}[H]
		\centering
		\subfigure[实验结果1] { \label{fig:b}
			\includemovie[
				autoplay,
				controls,
				repeat=3,
				text={\large(Loading video)}
			]{15cm}{8cm}{t_b1.mp4}
		}
	\end{figure}
	\begin{figure}[H]
		\centering
		\subfigure[实验结果2] { \label{fig:b}
			\includemovie[
				autoplay,
				controls,
				repeat=3,
				text={\large(Loading video)}
			]{15cm}{8cm}{t_b2.mp4}
		}
	\end{figure}
	不难发现,仅仅只是微小的位置差异,也能够让灰尘的运动轨迹大不相同,这说明三体问题中混沌现象确实是存在的。
	\subsection{机械波模拟实验}
	\subsubsection{实验原理}
	\subsubsection*{机械波模型的建立}
	在机械波问题中,我们采用类似于分子热运动学的模拟方法,在一维的线性弹簧振子链模型的基础上展开研究。在理想模型的设置上,我们参考了固体物理中一维晶格振动
	的理想模型,将传递介质看成是一维空间上无限多个质点的集合,并且假设质点在$x$轴上的位移始终为0。类似于一维晶格振动的纵波模型,我们假设:$u_n$为序号为$n$的
	质点在$t$时刻离开平衡位置的位移;设$r$为质点振动时序号为$n$和$n+1$的两原子间在$t$时刻的距离,$\delta$为两质点在$x$轴上的间距,那么:
	\[r=\sqrt{(u_{n+1} - u_n)^2+\delta^2}\]
	设$U(r)$为两原子之间的相互作用势,若假设原子的位置总在平衡位置附近,那么对$U(r)$作泰勒展开有:
	\[U(r) = U(\delta)+\sum_{k = 1}^{\infty} \frac{1}{k!}\frac{\mathrm{d}^kU}{{\mathrm{d}r}^k}(r-\delta)^k\]
	由此可得两原子之间的相互作用力为:
	\[f(r) = -\sum_{k = 1}^{\infty}\frac{1}{(k-1)!}\frac{\mathrm{d}^kU}{{\mathrm{d}r}^k}(r-\delta)^k\]
	令$\displaystyle k=\frac{\mathrm{d}^2U}{{\mathrm{d}r}^2}$,忽略掉其它的高阶无穷小量,我们得到第$n$个原子和第$n+1$个原子在$y$轴方向上的相互作用力大小为:
	\[f_n=k(u_{n+1}-u_n)-k(u_n-u_{n-1})=k(u_{n+1}+u_{n-1}-2u_n)\]
	由此可以得到第$n$个质点的运动方程:
	\[m\frac{\mathrm{d}^2U}{{\mathrm{d}r}^2}=k(u_{n+1}+u_{n-1}-2u_n)\]
	根据这个方程,我们就可以使用分子动力学方法进行模拟。
	\subsubsection*{驻波}
	两列振幅相同的相干波,在同一直线上沿相反方向传播时,互相叠加后将形成驻波。即驻波是一种特殊的干涉现象。我们依据机械波的基本模型,我们可以模拟驻波的实验过程。
	我们假设绳的一端固定,另一端受迫以一定的频率和振幅简谐振动,则一端振动产生的波与另一端产生的反射波恰好是一组振幅相同的相干波。\par
	假设正向的波动方程为:
	\[y=A\cos\left(\omega t+\frac{2\pi}{\lambda}+\phi_0\right)\]
	那么反射波的波动方程为:
	\[y=-A\cos\left(-\omega t+\frac{2\pi}{\lambda}+\phi_0\right)\]
	它们叠加后的波动方程为:
	\[y=2A\cos\left(\frac{2\pi}{\lambda}x\right)\cos(\omega t+\phi_0)\]
	因此我们得到的合成波应该有如下性质:
	\begin{itemize}
		\item 振幅为:
		\[A'=2A\cos\left(\frac{2\pi}{\lambda}\right)\]
		\item 波腹的位置为:
		\[x_0=\frac{\lambda}{2}k\]
		\item 波节的位置为:
		\[x_1=\frac{\lambda}{2}(2k+1)\]
	\end{itemize}
	\subsubsection{实验结果及分析}
	如下图为形成驻波之前的图像:
	\begin{figure}[H]
		\centering
		\includegraphics[height=3cm]{1.jpg}
		\caption{形成驻波之前的图像}
	\end{figure}
	下图为形成驻波之后的图像:
	\begin{figure}[H]
		\centering
		\includegraphics[height=3cm]{2.jpg}
		\caption{形成驻波之后的图像}
	\end{figure}
	值得注意的是,这里模拟的驻波并不稳定,我们猜想这是因为振源的频率设置差异的原因——即振源的频率设置不恰当,以及计算时的精度误差导致了该机械波不能够稳定
	存在,下图为不稳定时的图像:
	\begin{figure}[H]
		\centering
		\includegraphics[height=3cm]{3.jpg}
		\caption{不稳定时的图像}
	\end{figure}
	\newpage
	\section{附录与感想}
	\noindent 在这次研究过程中,我们有如下感想:
	\begin{enumerate}
		\item 分子动力学方法和事件驱动方法是两种不同的物理引擎实现方式,每种方式都各有其优缺点,但从整体来看,模拟的效果还是相当不错的,基本达到了预期效果;
		\item 物理引擎在研究一些需要大量计算或者不可求解析解的问题上有着非常大的作用,虽然计算结果不一定准确,但是这计算结果可以为以后真实的物理实验提供一个
		指导方向,从而节省不必要的人力物力;
		\item 通过这次的研究,我们也发现了OpenGL$^\text{TM}$的库的强大之处,在探究物理世界的同时也增进了编程水平;
		\item 感谢老师在研究过程中给予我们的关心与指导!
	\end{enumerate}
	本文所附带的实验代码均放置在作者的\href{https://github.com/PrayStarJirachi/OpenGL-Physical-Engine}{GitHub}上,其中通过其中的
	\href{https://github.com/PrayStarJirachi/OpenGL-Physical-Engine/tree/master/Execuable/Windows}{Windows版本的可执行文件}和
	\href{https://github.com/PrayStarJirachi/OpenGL-Physical-Engine/tree/master/Execuable/Linux}{Linux版本的可执行文件}可以得到上述的几个视频(当然初始化参数可能不同)。
	\newpage
	\bibliographystyle{plain}
	\bibliography{ref}
\end{document}

